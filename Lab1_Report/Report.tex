\documentclass{TechReport}
\usepackage{graphicx}
\usepackage{setspace}
\usepackage{lipsum}  
\usepackage{amsmath}
\usepackage{url}
\urlstyle{rm}
\usepackage{lscape}
\usepackage{booktabs}
\usepackage{multirow}
\usepackage[numbered,autolinebreaks,useliterate]{mcode}
\usepackage{listings}
\lstdefinestyle{nonumbers}
{numbers=none}
\usepackage{pdfpages}

% ----------------------------------------------------------------
\vfuzz2pt % Don't report over-full v-boxes if over-edge is small
\hfuzz2pt % Don't report over-full h-boxes if over-edge is small


%if you have installed and are using Windows' Times New Roman font, uncomment the following:
%\renewcommand{\encodingdefault}{T1}
%\renewcommand{\rmdefault}{tnr}

\newenvironment{packeditemize}%
{\begin{itemize}
  \setlength{\itemsep}{1pt}
  \setlength{\parskip}{0pt}
  \setlength{\parsep}{0pt}
}%
{\end{itemize}}

\setstretch{0.9} 


\begin{document}
%
\title{ELEN4020: Data and Intensive Computing Laboratory Exercise 1}
%
\author{Timothy McBride (732037) \hspace{1.5em} Nabeel Vandayar (704528) \hspace{1.5em} Dane Slattery 789132\\  
\\	
\emph{University of the Witwatersrand, School of Electrical and Information Engineering,\\ Private Bag 3, 2050,
	Johannesburg, South Africa
                                                              }}

\abstract{Three procedures were created for an array of K dimensions the procedures use a one dimensional array to store the elements and use mathematical algorithms to dimensionally index the elements. The three procedures were required to operate on four different arrays and were implemented in the C programming language.}

\keywords{Open Mp, array, time complexity}
%
\maketitle
%
% ----------------------------------------------------------------
\section{Introduction}
\label{sec:Introduction}
This main focus for this laboratory was to create a methodology to perform operations on multidimensional arrays of varying size. The exercise makes use of the C programming language to create three procedures that operate on K dimensional arrays with an N time complexity. These three procedures take as input; a K dimensional integral array, the dimensional bounds of that array and an integer for the total number of dimensions within the array. The purpose of the procedures are:
\begin{enumerate}
\item Set all elements in the array to zero.
\item Set ten percent of the elements in the array uniformly to one.
\item Select five percent of the elements in the array and then display their coordinates and value.
\end{enumerate} 
Additionally a main program is written to create four arrays, the procedures that are developed are run on each of these arrays.
 
 \section{Threading comparison}
 POSIX Threads

    Specified by the IEEE POSIX 1003.1c standard (1995). C Language only.
    Part of Unix/Linux operating systems
    Library based
    Commonly referred to as Pthreads.
    Very explicit parallelism; requires significant programmer attention to detail. 

OpenMP

    Industry standard, jointly defined and endorsed by a group of major computer hardware and software vendors, organizations and 	individuals.
    Compiler directive based
    Portable / multi-platform, including Unix and Windows platforms
    Available in C/C++ and Fortran implementations
    Can be very easy and simple to use - provides for "incremental parallelism". Can begin with serial code. 
 
\section{Problem Solution}
This problem was solved by operating on a one dimensional array which is mathematically manipulated to act as a multidimensional array.  The array is created by allocating the equivalent memory required for the two dimensional array. The procedures are designed to take the one dimensional array as input and operate on it accordingly. 

\subsection{Procedure 1}
Procedure one uses a standard loop to iterate through each element of the array and then sets the specific element to zero.

\subsection{Procedure 2}
Procedure two determines ten percent of the total elements of the array, this number is rounded off if it is a fraction. A bias is calculated to ensure that each of the ones generated is uniformly distributed across the array. A loop is used to assign a one to the element at the desired location determined by the bias.

\subsection{Procedure 3}
Procedure three determines the number of elements approximately equivalent to five percent of the total array. This value is used to iterate over a loop that many times. The program selects this many elements and determines the dimensional coordinates for the element. The coordinates and the value contained by the element are printed to the screen.


\section{Conclusion}
The problems assigned for the laboratory were solved using a one dimensional array. This method proved effective as iteration through the array has a maximum time complexity of n where n is the number of elements in the array. The problems were solved according to the specifications however the parameters for each of the procedures required an additional input, this was the number of dimensions within the bounds array.  



\end{document}
% ----------------------------------------------------------------
