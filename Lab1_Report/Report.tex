\documentclass{TechReport}
\usepackage{graphicx}
\usepackage{setspace}
\usepackage{lipsum}  
\usepackage{amsmath}
\usepackage{url}
\urlstyle{rm}
\usepackage{lscape}
\usepackage{booktabs}
\usepackage{multirow}
\usepackage[numbered,autolinebreaks,useliterate]{mcode}
\usepackage{listings}
\lstdefinestyle{nonumbers}
{numbers=none}
\usepackage{pdfpages}

% ----------------------------------------------------------------
\vfuzz2pt % Don't report over-full v-boxes if over-edge is small
\hfuzz2pt % Don't report over-full h-boxes if over-edge is small


%if you have installed and are using Windows' Times New Roman font, uncomment the following:
%\renewcommand{\encodingdefault}{T1}
%\renewcommand{\rmdefault}{tnr}

\newenvironment{packeditemize}%
{\begin{itemize}
  \setlength{\itemsep}{1pt}
  \setlength{\parskip}{0pt}
  \setlength{\parsep}{0pt}
}%
{\end{itemize}}

\setstretch{0.9} 


\begin{document}
%
\title{ELEN4020: Data and Intensive Computing Laboratory Exercise 1}
%
\author{Timothy McBride (732037) \hspace{1.5em} Nabeel Vandayar (704528) \hspace{1.5em} Dane Slattery 789132\\  
\\	
\emph{University of the Witwatersrand, School of Electrical and Information Engineering,\\ Private Bag 3, 2050,
	Johannesburg, South Africa
                                                              }}

\abstract{The abstract}

\keywords{keywords}
%
\maketitle
%
% ----------------------------------------------------------------
\section{Introduction}
\label{sec:Introduction}
This main focus for this laboratory was to create a methodology to perform operations on multidimensional arrays of varying size. The exercise makes use of the C programming language to create three procedures that operate on K dimensional arrays. These three procedures take as input; a K dimensional integral array, the dimensional bounds of that array and an integer for the total number of dimensions within the array. The purpose of the procedures are:
\begin{enumerate}
\item Set all elements in the array to zero.
\item Set ten percent of the elements in the array uniformly to one.
\item Select five percent of the elements in the array and then display their coordinates and value.
\end{enumerate} 
Additionally a main program is written to create four arrays, the procedures that are developed are run on each of these arrays.
 
\section{Problem Solution}
This problem was solved by operating on a one dimensional array which is mathematically manipulated to act as a multidimensional array. This allows for k dimensional arrays to be created, for the procedures. 

\section{Instructions to Access the Repository}

\section{Conclusion}
\label{sec:Conclusion}


% References
\begin{thebibliography}{99}	
%	include data sheets
\bibitem{ref:Nabs}



	 
\end{thebibliography}

\end{document}
% ----------------------------------------------------------------
